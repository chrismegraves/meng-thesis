\chapter{Case Projects}
It would be impossible to properly analyze the merits and faults of a variety of multi-touch interaction set designs without first considering a few of the most typical, potential use cases. In this chapter, I will present six project ideas that are representative of the kinds of projects that Tablet Scratchers of varying experience levels are likely to want to create. These six project ideas can be best thought of as three iterations of two basic projects: a Pong-style project and a finger-painting project. The three iterations are increasing in complexity and represent desirable projects for Scratchers at three different experience levels. I categorize the three iterations as \emph{low floor cases}, \emph{middle height cases}, and \emph{high ceiling cases}. The low floor cases represent simple, basic projects that even novice Scratchers should be able to intuitively create with little or no prior experience. Meanwhile, the middle height cases represent slightly more complicated additions that ought to be implementable by a Scratcher with significant experience. Finally, the high ceiling cases represent augmentations that an expert Scratcher might aspire to make. Together, these projects will serve as benchmarks for which I evaluate the multi-touch interaction set designs that I will present and analyze in the following three chapters.

\subsubsection{Introduction to Pong}
Designed by Allan Alcorn in 1972, Pong is widely considered to be the first culturally significant video game. Although rather simple in its original incarnation, many more complex variations have since been created.  In fact, Pong-style games are quite popular within the Desktop Scratch community, as evidenced by the thousands of variants that Scratchers have created and shared. Pong is a particularly fitting project to demonstrate the floors and ceilings of the multi-touch interaction set designs, since Pong projects can begin with very simple touch interactions that later become complex through the addition of more users and new actions.

\subsubsection{Introduction to Finger Painting}
Since the early 1900s, children all over the world have been expressing themselves through the art of finger painting. For many, painting with one's fingers is a calming, liberating experience. Due to their capacity for touch interactivity, tablet computers naturally lend themselves to being a means for finger painting, albeit cleaner and less textured. Like Pong, finger painting is another project that will help with presenting the floors and ceilings of the multi-touch interaction set designs. Although implementing painting for just one finger at a time is rather simple, the complexity increases dramatically as support for more fingers is added.

\section{Low Floor Cases}
It is worth restating that it is of paramount importance that the multi-touch interaction set be designed with a low floor in mind. Novice Tablet Scratchers must be able to easily figure out how to use the multi-touch interaction set to create simple projects that are responsive to touch. If beginners are able to create simple touch interactions without much difficulty, their early success will encourage them to augment their touch interactions and continue tinkering with their projects. Otherwise, if the process of creating simple touch-interactive projects is painful, many beginners will become frustrated and will forgo improving their projects. Here I specify two examples of basic projects that new Scratchers should be able to create without much assistance or hardship.

\subsection{Basic One-Player Pong}
The simplest form of Pong consists of a single paddle and a ball, which moves around the \emph{stage} (i.e., the background of a Scratch project) at a constant speed, rebounding whenever it hits either the paddle or a wall (Figure \ref{OnePlayerPong}). The paddle has a fixed y-coordinate (and thus cannot be moved up or down), but can be moved left and right. To begin any Pong project, one must first implement this basic functionality.

\begin{figure}
\centering
\includegraphics[width=0.6\textwidth]{images/OnePlayerPong.PNG}
\caption[One-Player Pong Screenshot]
{Here is a screenshot from a one-player Pong game.}
\label{OnePlayerPong}
\end{figure}

This single-player form of Pong has for a long time been one of the example starter projects in Desktop Scratch. In the starter project, the x-coordinate of the paddle is simply mapped to that of the mouse cursor. Therefore, users control the paddle by moving the mouse left and right. Note that in this case, the mouse controls the movement of the paddle regardless of whether or not the mouse cursor is hovering over the paddle. Unfortunately, we cannot use this control scheme for a Tablet Scratch project, since tablets do not have mice.

For this case project, I will specify an analogous touch control scheme for tablets. In the Tablet Scratch equivalent of this starter project, the paddle's x-coordinate will begin at an arbitrary default state. When a user touches the project, the paddle's x-coordinate will follow the x-coordinate of the user's touch. After the touch is lifted, the paddle's x-coordinate can either remain where it is or return to the arbitrary default state. 

The purpose of this case project is to test how easily novice Scratchers can create a single-finger touch interaction that relies on extracting coordinate information from a touch (in this case the x-coordinate) using the various multi-touch interaction sets. As a result, it will be assumed for this project that the user never presses two or more fingers on the tablet at the same time and I will leave it unspecified how the implementations of this project should handle such simultaneous touches.

\subsection{One-Finger Painting}
In the same way that one-player Pong is the simplest form of interactive Pong, a one-finger painting project is the most basic form of a finger-painting application. At its simplest, a one-finger painting application starts with a blank stage and paints an arbitrary color wherever the user touches the screen with one finger at a time (Figure \ref{OneFingerPainting}). Again, it will be assumed for this project that the user will not touch the screen with more than one finger at a time, so I will leave it unspecified how implementations should handle cases when the assumption is broken.

\begin{figure}
\centering
\includegraphics[width=0.6\textwidth]{images/OneFingerPainting.PNG}
\caption[One-Finger Painting Screenshot]
{Here is a sample painting made using a one-finger painting project.}
\label{OneFingerPainting}
\end{figure}

In Desktop Scratch, an analogue of this project can be created easily by using a \emph{pen}. As an homage to its predecessor, Logo, each sprite in Scratch carries a pen, which, when placed down, traces the movement of the sprite onto the stage. Consequently, a Scratcher would begin a ``mouse painting'' project in Desktop Scratch by first creating a \emph{paintbrush} sprite that follows the mouse cursor wherever it goes. Next, the Scratcher would program the paintbrush sprite's pen to start in the up position, but go down only when the mouse clicks down (meaning the pen only traces the movement of the mouse when the mouse is clicked down).

For the Tablet Scratch version of the project, the Scratcher would want the paintbrush sprite to initially have its pen up. Whenever a finger is pressed down, the sprite must be programmed to first move to the touch location, then put its pen down, and finally follow the touch until it is lifted. At the moment the touch is lifted, the sprite must be programmed to lift up its pen. Accordingly, the purpose of this case project is to demonstrate how easily a novice Scratcher can code a sprite to follow a touch and react to when the touch begins and ends.
 
At its core, this project is a simple exercise of recognizing and handling when a single finger is pressed down, when it moves, and when it is lifted. This capability is the building block to coding virtually all single-finger touch interactions. Thus, an interaction set that allows novice Tablet Scratchers to easily figure out how to implement this case project would be quite powerful.

\section{Middle Height Cases}
As Tablet Scratchers gain experience, they will naturally want to augment their simpler projects to develop progressively more complex ones. In addition to being intrinsically rewarding, the process of building upon earlier projects is a fundamental part of constructionist learning. Therefore, the ideal multi-touch interaction set must be designed so that as Tablet Scratchers master the basics of simple touch interactions, they can continue playing and figure out how to make more complex ones. If, on the other hand, the multi-touch interaction set is designed such that touch interactions more complex than the low floor cases are terribly difficult to achieve, then Tablet Scratchers would be stuck developing simple touch interactions and thus robbed of a learning opportunity. Here I present two natural augmentations an experienced Tablet Scratcher might want (and should be able) to add to the two previously described low floor cases. 

\subsection{Basic Two-Player Pong}
After a Tablet Scratcher successfully creates a simple one-player Pong project, the logical next step is to add a second player. For this case project, we again have a ball which moves around the stage with a constant speed, rebounding when it either hits a wall or a paddle. However, instead of one paddle with a fixed y-coordinate that moves only left and right, there are now, on either side of the stage, two paddles with fixed x-coordinates that move only up and down (Figure \ref{TwoPlayerPong}). Note that unlike in the one-player Pong project, this interaction cannot be developed in Desktop Scratch due to being limited to a single mouse.

\begin{figure}
\centering
\includegraphics[width=0.6\textwidth]{images/TwoPlayerPong.PNG}
\caption[Two-Player Pong Screenshot]
{Here is a screenshot from a two-player Pong game.}
\label{TwoPlayerPong}
\end{figure}

Clearly the controls from the one-player Pong project will not work in this case, because there could only be one touch at a time and that touch would control both paddles. For this case, I will assume that the paddles only follow the y-coordinate of a touch if the location of the touch is within a defined control range of the paddle and that the control range is small enough that no touch can simultaneously control both paddles. To make things simpler, it will be assumed that there are never more than two simultaneous touches and there is at most one touch at a time within either paddle's control range. 

The purpose of this case project is to compare the relative ease of using the various multi-touch interaction set designs to develop projects that require multiple sprites to respond to the touches that are closest to them. Without this functionality, many (even simple) multi-finger interactions will be impossible to develop. Although the capability for sprites to respond to all touches on the screen is obviously more powerful, an extensive range of the most common multi-touch interactions can be developed when the sprites' ability to respond to touches is limited to the ones that are closest to them.

\subsection{Two-Finger Painting}
Similar to how two-player Pong is the natural next step after one-player Pong, two-finger painting is the natural next step after one-finger painting. In two-finger painting, the user is able to draw on the stage with two fingers simultaneously instead of being limited to just one finger at a time (Figure \ref{TwoFingerPainting}). Again for simplicity, it will be assumed that the user never has more than two fingers simultaneously pressed on the tablet. Also note that, again, this kind of interaction cannot be developed for a single mouse in Desktop Scratch.

\begin{figure}
\centering
\includegraphics[width=0.6\textwidth]{images/TwoFingerPainting.PNG}
\caption[Two-Finger Painting Screenshot]
{Here is a sample painting made using a two-finger painting project.}
\label{TwoFingerPainting}
\end{figure}

This appended functionality adds a significant amount of complexity to the project. Now instead of having just one paintbrush sprite, there necessarily must be two paintbrush sprites to follow the up to two fingers pressed on the tablet. Moreover, when two fingers are being pressed at the same time, these two paintbrush sprites must be able to distribute themselves amongst the fingers, so that both fingers have a following paintbrush sprite. 

In contrast to the two-player Pong game, it is not enough in this project for the paintbrush sprites to simply follow the closest touch. If the paintbrush sprites were to do so, they would be prone to end up following the same finger, leaving the other finger without a paintbrush. There are many potential projects where it is necessary for sprites to distribute themselves among touches. Programming sprites to always follow separate fingers is a challenge, so the purpose of this project is to help distinguish which of the multi-touch interaction set designs make it easier.  

\section{High Ceiling Cases}
While Scratch's mission prioritizes having a low floor and wide walls, its high ceiling has been a major factor in its success. There are certain complex tasks, such as creating 3D graphics and online multiplayer capabilities, that are particularly difficult to develop in Scratch, but are possible to implement. Every day a few expert Scratchers implement and share projects that continue to push the bounds of what is possible to create in Scratch. Not only do the expert Scratchers gain a valuable learning experience by coding these challenging projects, but by sharing them with the community they inspire and educate Scratch beginners. By playing with the high ceiling projects and looking at how they were executed, less experienced Scratchers are motivated to continue challenging themselves and exploring what is possible. Thus, while complex multi-touch interactions might not necessarily be easy to program in Tablet Scratch, they should at least be possible to develop. Here I present two projects which feature advanced augmentations to the middle height case projects.

\subsection{Advanced Two-Player Pong}
While a basic two-player Pong game can be fun to play for a while, it is quite simple and eventually gets boring. The ball moves at a constant speed and players are limited to just moving their paddle up and down. In an attempt to make the game more interesting, an expert Scratcher might try adding a \emph{bump} action. A bump action causes the paddle to jut forward towards the opponent for a half-second before returning to its previously fixed x-coordinate. If the paddle hits the ball when it is in the process of a bump, the ball's speed increases by a constant factor. The bump action adds a little more skill and excitement to the basic two-player Pong project, but the question that remains is how will users instigate a bump action.

For this case project, a player will bump their paddle each time a second touch is pressed within a paddle's control range while there is already a touch there. More simply put, albeit slightly less rigorously, a bump happens whenever a user already using one finger to move the paddle presses their second finger close to the paddle.

At the core of this interaction is the capacity for sprites to recognize and handle not only the touch closest to them, but also all subsequent nearby touches. Although this interaction feels natural for this particular project case, it is very specific and unconventional. The ability to design and implement complex, multi-touch gestures adds a powerful means of personalizing Tablet Scratch projects. Hence, the ideal multi-touch interaction set design will make it possible to develop complex multi-touch controls like the one exhibited in this project.

\subsection{Ten-Finger Painting}
The next step for the finger-painting project after adding support for a second finger, is to add support for even more fingers. Since many Android devices can recognize up to ten simultaneous touches (not to mention that only a few people have more than ten fingers), this case project will be a finger painting application that supports up to ten simultaneous finger touches (Figure \ref{TenFingerPainting}). Many expert Tablet Scratchers will naturally try to make projects that allow users to deploy as many fingers as possible. Thus, the purpose of this case project is to test how well the multi-touch interaction sets make it possible to handle a large number of simultaneous touches. 

\begin{figure}
\centering
\includegraphics[width=0.6\textwidth]{images/TenFingerPainting.PNG}
\caption[Ten-Finger Painting Screenshot]
{Here is a sample painting made using a ten-finger painting project.}
\label{TenFingerPainting}
\end{figure}

Notice that the problems for this project are essentially the same as those regarding the two-finger painting project, but scaled by a factor of five. While the similarities might make this case project appear redundant, certain multi-touch interaction set designs make it easy to handle one or two fingers but terribly difficult to handle even a few more fingers. Meanwhile, other multi-touch interaction set designs make it a little challenging to handle one or two fingers, but not any more challenging to handle ten. I will discuss these trade-offs in the following three chapter as I introduce and analyze a selection of multi-touch interaction set designs.
