\chapter{Conclusion}
\section{Closing Remarks}
\section{Future Work}


%The more complicated of the new blocks is the ``when touch created" block. This block is a completely original kind of block that can be best described as cross between a trigger block and a definition block. Much like the ``when finger pressed" block from the extended version of the Closest Finger Design, the ``when touch created" block creates a clone of its connected stack and executes it in a separate thread. However, differing from the ``when finger pressed block", the ``when touch created" does not directly change the touch of interest for the touch blocks within its stack depending on the triggering touch. Instead, much like a definition block, the ``when touch created" block has a ``created touch ID" block within it that can be used throughout the execution of the stack, but only within the stack. Thus, in each thread that is spawned from the ``when touch created" block, the ``created touch ID" takes the value of the touch ID that triggered creation of the thread.

%In many ways, this ``when touch created" block is a callback function since it it is called every time an event happens with the triggering event given as a parameter value. However, this block avoids most of the problems that plague callback functions (some of which I described in Chapter 2), by producing a new thread for the handling of each event. Unfortunately, by having this thread spawning feature, the ``when touch created" is susceptible  many of the understandability issues that the ``when finger pressed" block suffers from. 

