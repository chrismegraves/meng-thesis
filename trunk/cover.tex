% -*-latex-*-

\title{Scratching with All Your Fingers: Exploring Multi-Touch Programming in Scratch}

\author{Christopher Graves}
\prevdegrees{ S.B., Massachusetts Institute of Technology (2013)}
% If you wish to list your previous degrees on the cover page, use the 
% previous degrees command:
%       \prevdegrees{A.A., Harvard University (1985)}
% You can use the \\ command to list multiple previous degrees
%       \prevdegrees{B.S., University of California (1978) \\
%                    S.M., Massachusetts Institute of Technology (1981)}
\department{Department of Electrical Engineering and Computer Science}

% If the thesis is for two degrees simultaneously, list them both
% separated by \and like this:
% \degree{Doctor of Philosophy \and Master of Science}
\degree{Master of Engineering in Electrical Engineering and Computer Science}

% As of the 2007-08 academic year, valid degree months are September, 
% February, or June.  The default is June.
\degreemonth{June}
\degreeyear{2014}
\thesisdate{May 23, 2014}

%% By default, the thesis will be copyrighted to MIT.  If you need to copyright
%% the thesis to yourself, just specify the `vi' documentclass option.  If for
%% some reason you want to exactly specify the copyright notice text, you can
%% use the \copyrightnoticetext command.  


% If there is more than one supervisor, use the \supervisor command
% once for each.
\supervisor{Prof. Mitchel Resnick}{LEGO Papert Professor of Learning Research, MIT Media Lab}

% This is the department committee chairman, not the thesis committee
% chairman.  You should replace this with your Department's Committee
% Chairman.
\chairman{Prof. Albert R. Meyer}{Chairman, Masters of Engineering Thesis Committee}

% Make the titlepage based on the above information.  If you need
% something special and can't use the standard form, you can specify
% the exact text of the titlepage yourself.  Put it in a titlepage
% environment and leave blank lines where you want vertical space.
% The spaces will be adjusted to fill the entire page.  The dotted
% lines for the signatures are made with the \signature command.
\maketitle

% The abstractpage environment sets up everything on the page except
% the text itself.  The title and other header material are put at the
% top of the page, and the supervisors are listed at the bottom.  A
% new page is begun both before and after.  Of course, an abstract may
% be more than one page itself.  If you need more control over the
% format of the page, you can use the abstract environment, which puts
% the word "Abstract" at the beginning and single spaces its text.

%% You can either \input (*not* \include) your abstract file, or you can put
%% the text of the abstract directly between the \begin{abstractpage} and
%% \end{abstractpage} commands.

% First copy: start a new page, and save the page number.
\cleardoublepage
% Uncomment the next line if you do NOT want a page number on your
% abstract and acknowledgments pages.
% \pagestyle{empty}
\setcounter{savepage}{\thepage}
\begin{abstractpage}
% $Log: abstract.tex,v $
% Revision 1.1  93/05/14  14:56:25  starflt
% Initial revision
% 
% Revision 1.1  90/05/04  10:41:01  lwvanels
% Initial revision
% 
%
%% The text of your abstract and nothing else (other than comments) goes here.
%% It will be single-spaced and the rest of the text that is supposed to go on
%% the abstract page will be generated by the abstractpage environment.  This
%% file should be \input (not \include 'd) from cover.tex.

Since the introduction of the iPhone in 2007, many millions of people have used a multi-touch interface; but, due to the inaccessibility of most tablet software development kits, very few of these people have ever developed their own multi-touch interactions. This thesis discusses the challenges in developing a toolkit that allows novices to easily make simple touch-interactive projects, while simultaneously empowering experienced users to create complex, personalized multi-touch interactions. Three potential toolkit designs are presented and evaluated using the principle of ``low floors, wide walls, and high ceilings." The toolkits presented have been developed within the context of an upcoming tablet version of Scratch, which aims to allow users of all ages and educational backgrounds (but school-aged children in particular) to easily make and share their own stories, games, and animations on and for the tablet.
\end{abstractpage}

% Additional copy: start a new page, and reset the page number.  This way,
% the second copy of the abstract is not counted as separate pages.
% Uncomment the next 6 lines if you need two copies of the abstract
% page.
% \setcounter{page}{\thesavepage}
% \begin{abstractpage}
% % $Log: abstract.tex,v $
% Revision 1.1  93/05/14  14:56:25  starflt
% Initial revision
% 
% Revision 1.1  90/05/04  10:41:01  lwvanels
% Initial revision
% 
%
%% The text of your abstract and nothing else (other than comments) goes here.
%% It will be single-spaced and the rest of the text that is supposed to go on
%% the abstract page will be generated by the abstractpage environment.  This
%% file should be \input (not \include 'd) from cover.tex.

Since the introduction of the iPhone in 2007, many millions of people have used a multi-touch interface; but, due to the inaccessibility of most tablet software development kits, very few of these people have ever developed their own multi-touch interactions. This thesis discusses the challenges in developing a toolkit that allows novices to easily make simple touch-interactive projects, while simultaneously empowering experienced users to create complex, personalized multi-touch interactions. Three potential toolkit designs are presented and evaluated using the principle of ``low floors, wide walls, and high ceilings." The toolkits presented have been developed within the context of an upcoming tablet version of Scratch, which aims to allow users of all ages and educational backgrounds (but school-aged children in particular) to easily make and share their own stories, games, and animations on and for the tablet.
% \end{abstractpage}

\cleardoublepage

\section*{Acknowledgments}

This is the acknowledgments section.  You should replace this with your
own acknowledgments.

%%%%%%%%%%%%%%%%%%%%%%%%%%%%%%%%%%%%%%%%%%%%%%%%%%%%%%%%%%%%%%%%%%%%%%
% -*-latex-*-
