\chapter{Conclusion}

In this final chapter, I will begin by summarizing my analysis of the multi-touch interaction set designs presented and conclude by making suggestions for how this research may be furthered.

\section{Closing Remarks}

Even though each of the three multi-touch interaction set designs has its strengths, none of them fully satisfies the goals for maintaining low floors, wide walls, and high ceilings. While the minimalism of the Closest Finger Design  makes developing simple touch-interactivity accessible to beginners, it also precludes more advanced users from creating more complex interactions. In contrast, the Absolute Indexing Design is challenging for beginners to understand but, once understood, can be used to create limitlessly intricate multi-touch interactions. Meanwhile, the Relative Indexing Design sits uncomfortably in the middle, by being somewhat difficult to understand while still not being powerful enough to be used to implement even moderately complex multi-touch interactions.

Although I have presented these three multi-touch interaction set designs in isolation (in order to highlight their individual strengths and weaknesses), the best multi-touch interaction set is, arguably, most likely a combination of either two or all three of them. Ideally, Tablet Scratch's multi-touch interaction set will have the low floors of the Closest Finger Design and the high ceilings of the Absolute Indexing Design. However, it would be ill-advised to simply take the union of the two interaction set designs due to the confusion the considerable overlap would cause.

One way to combine the Closest Finger and Absolute Indexing designs would be to add both interaction sets to Tablet Scratch, but to feature them in separate locations. For example, Tablet Scratch could be designed so that sprites can only use the Closest Finger Design blocks and the stage can only use the Absolute Indexing Design blocks. This compromise is satisfactory on several levels. First, the closest finger is often not important for the stage in the same way that is for sprites. Furthermore, the Scratchers who assemble scripts for the stage are generally experienced and desire to make complex projects. Hence, only the Scratchers who need the more powerful blocks would be exposed to them.

While this combination preserves the Closest Finger Design's low floors, it does not quite preserve the Absolute Indexing Design's high ceilings. Since each sprite in this scenario can only access its closest touch, other touch information must be transferred from the stage to the sprite. Scratch does support such data transferring, but in certain scenarios it can be very difficult to implement robustly.

\section{Future Work}

Since low floors are generally achieved at the cost of high ceilings (and vice versa), it is unlikely that an indisputably ideal interaction set can ever be achieved. However, through more iteration and testing, we can get closer to designing a multi-touch interaction set that is perfect for the purposes of Tablet Scratch.

One important next step to reach this goal is to have the target audiences test the various multi-touch interaction set designs and then to iterate based on their experiences. Each potential interaction set design should be tested with Scratchers of varying experience levels to ensure that the design meets computer scientist Alan Kay's requirement that ``simple things should be simple, complex things should be possible."

Although this thesis has focused on creating a multi-touch interaction set specifically within the context of Scratch, many of the ideas discussed are also applicable to designing touch-input-handling toolkits in general. There is little doubt that the toolkits used today in ``professional frameworks" are less than ideal. Not only do novice programmers have difficulty understanding how to use these toolkits, but even expert software engineers are prone to making errors while using them to develop multi-touch interactions. Despite the fact that virtually all commonly used multi-touch interactions have been developed with these professional frameworks, I am confident that it is possible to develop an equally functional multi-touch toolkit that is both more accessible and less vulnerable to bugs.

